\documentclass[a4paper, 10pt]{scrbook}

\usepackage[style=alphabetic, backend=biber, isbn=false, doi=false, maxcitenames=2, autolang=langname, maxbibnames=99]{biblatex}
\addbibresource{literature.bib}
%%% No pagebreak before bibliography:
\defbibheading{secbib}[\bibname]{%
  \section*{#1}%
  \markboth{#1}{#1}}
%%%

\setcounter{secnumdepth}{3}
%%% Number width in toc
\usepackage{tocloft}
\renewcommand\cftpartnumwidth{1.5cm}
%%% Seperate ToC for appendix:
\usepackage{etoc}
%%%
% \usepackage{helvet}
\usepackage[sfdefault]{andika}
% \usepackage[
% %   italic,
%     symbolgreek
%     ]{mathastext}
% \usepackage[bb=px]{mathalpha}

\makeatletter
\renewcommand{\paragraph}{%
  \@startsection{paragraph}{4}%
  {\z@}{1.75ex \@plus 1ex \@minus .2ex}{-1em}%
  {\sffamily\small\bfseries}%
}
\makeatother

\usepackage[table,xcdraw]{xcolor}
\usepackage{schule}

\usepackage{array}
\usepackage{tabularx}
\newcommand{\rowheight}{\parbox[c][1cm][s]{0.1cm}{~}}
\newcommand{\rowheightTable}{\parbox[c][2cm][s]{0.1cm}{~}}

\usepackage[shortlabels]{enumitem}
\newlist{todolist}{itemize}{2}
\setlist[todolist]{label=$\square$}
\usepackage{pdflscape}
\usepackage{pdfpages}

\usepackage{wrapfig}
\usepackage[font=small,labelfont=bf, format=plain]{caption}

\usepackage{hyperref}
\usepackage{geometry}
\geometry{margin=3cm}
\usepackage{multicol}
\usepackage{amsmath, amssymb}
\let\oldemptyset\emptyset
\let\emptyset\varnothing

\usepackage{amsthm}
\newtheorem*{thm}{Satz}
\newtheorem*{defn}{Definition}
\newtheorem*{claim}{Behauptung}
\newtheorem*{ex}{Beispiel}
\newtheorem*{caution}{Achtung}
\newtheorem*{exercise}{Aufgabe(n)}

\usepackage{tikz}
\usetikzlibrary{shapes,snakes}
\usetikzlibrary{arrows.meta}
\usetikzlibrary{calc,patterns,angles,quotes}

\newcommand{\kariert}[3][]{
	\begin{tikzpicture}
	\draw[step=0.5cm,color=gray!50] (0,0) grid (#2 cm ,#3 cm);
	\node[anchor=north west] at (0,#3-.5) {#1};
	\end{tikzpicture}}
\newcommand{\kariertTop}[3][]{
	\begin{tikzpicture}
	\draw[step=0.5cm,color=gray!50] (0,0) grid (#2 cm ,#3 cm);
	\node[anchor=north west] at (0,#3) {#1};
	\end{tikzpicture}}
	
\tikzset{mygrid/.style={line width=1pt, dash pattern=on 0mm off 5mm, line cap=round}}
\newcommand{\dotted}[2]{
	\begin{tikzpicture}
        % \draw[step=0.5cm,color=gray,mygrid] (0,0) grid (#1,#2);
        \draw[step=0.5cm,color=gray,mygrid] (0,0) grid (#1 cm,#2 cm);
    \end{tikzpicture}
    }
    
\usepackage[ngerman]{babel}

\newcommand{\checked}{\makebox[0pt][l]{$\square$}\raisebox{.15ex}{\hspace{0.1em}$\checkmark$}}


\renewcommand{\vec}[2]{\begin{pmatrix}
           #1 \\
           #2 
         \end{pmatrix}}
         
\newcommand{\vc}[1]{\overrightarrow{#1}}

\newcommand{\highlight}{\bfseries\color{blue!50!black}}

\newcommand\myfbox[2]{\fcolorbox{white}{#1}{#2}}

\newcommand\el{\text{e}}

\title{PAM}
\subtitle{25Mz -- HS 24/25}
\author{Franziska Flegel}
\date{~}

\begin{document}
\frontmatter
\maketitle

\etocdepthtag.toc{mtchapter}
\etocsettagdepth{mtchapter}{subsection}
\etocsettagdepth{mtappendix}{none}
\tableofcontents

\etoctocstyle{1}{Anhang}
\etocdepthtag.toc{mtchapter}
\etocsettagdepth{mtchapter}{none}
\etocsettagdepth{mtappendix}{chapter}
\tableofcontents

\mainmatter

\chapter{Einführung in die Quantenphysik}
\section{Quantelung des Lichts}
\subsection{Hallwachs-Versuch}
Siehe Material zur Stunde am Dienstag, den 05.11.2024.\\
Literatur: Dorn-Bader SII Gesamtband Gymnasium 2023, S. 304--305 \cite{DornBader2023} 
\subsection{Bestimmung von $h$ mit der Gegenfeldmethode}
Siehe Material zur Stunde am Mittwoch, den 06.11.2024.\\
Simulation:\\~\\
  \begin{minipage}[t]{.15\linewidth}
  ~\\\vspace{-1cm}
  \includegraphics[width=\linewidth]{images/gym-wst-QR}
 \end{minipage}
 \hfill
 \begin{minipage}[t]{.8\linewidth}
  \url{https://www.physik.gym-wst.de/localhtml/qp/02_photoeffekt02/photoeffekt02.html}\\
 \end{minipage}
\subsection{Umkehrung des Fotoeffekts mit LEDs}
Siehe Material zur Stunde am Donnerstag, den 07.11.2024.\\
Literatur: Metzler Physik, 4. Auflage, S. 381 \cite{Metzler2007}.
\subsection{Kurzwellige Grenze von Röntgenlicht}
Siehe Material zur Stunde am Donnerstag, den 07.11.2024.\\
Literatur: Metzler Physik, 4. Auflage, S. 382--383 \cite{Metzler2007}.

\subsection{Compton-Effekt}

Bei seinem Experiment beobachtete Compton in Abhängigkeit vom Streuwinkel $\theta$ die Spektren Abbildung \ref{fig:cpt:spektren}.
Diese Beobachtung lässt sich nicht mit dem Wellencharakter des Lichts erklären.
Nimmt man jedoch einen Teilchencharakter an, so kann man folgende Beziehung zwischen der Wellenlängenänderung $\Delta \lambda$ und dem Streuwinkel $\theta$ herleiten:
\begin{align}
 \Delta \lambda = \lambda' - \lambda = \frac{h}{m_{\text{0\el}} c}\, \left( 1 - \cos\theta \right)
 \label{equ:cpt:ComptonFormel}
\end{align}
Die Herleitung dieser Formel ist in Abschnitt \ref{sec:cpt:herleitung} auf Seite \pageref{sec:cpt:herleitung} nachzulesen.

Eine gute Übersicht über die Geschichte des Compton-Effekts liefert weiterhin das Video \cite{youtubeComptonEng}.
Die mathematische Herleitung der obigen Formel gefällt mir persönlich aber in dem Video \cite{youtubeComptonDeu} besser.

\subsubsection{Compton-Wellenlänge}

Um sich einen Eindruck über die Größenordnung der Wellenlängenänderung zu verschaffen, hat man die sogenannte {\bfseries Compton-Wellenlänge} $\lambda_\text{C}$ als die Wellenlängenänderung bei $\theta = 90^\circ$ festgelegt:
\begin{align}
  \Delta \lambda_{90^\circ} = \lambda_\text{C} = 2.43\,\cdot\, 10^{-12}\,\text{m}
  = 2.43\,\text{pm}
\end{align}


% \subsubsection{Versuchsbeobachtung}
\begin{figure}
\begin{center}
  \includegraphics[width=.65\linewidth]{images/compdat.jpg}
\end{center}
\caption{Beim Compton-Effekt beobachtete Zählraten in Abhängigkeit vom Streuwinkel $\theta$.
Bildquelle: Website \cite{CompDat}.
\label{fig:cpt:spektren}}
\end{figure}

\begin{figure} 
\centering
\begin{tikzpicture}
\draw [->] (0,0) -- (0,3,0) node [above] {$y$};
\draw [black, thick ,-{Latex[length=4mm]},decorate, decoration={snake,amplitude=1mm,segment length=4mm,post length=4mm}, label=left:Incident photon] (-100pt,0pt) node[above]{Röntgenlicht mit Wellenlänge $\lambda$} -- (0,0pt);
\draw [black, thick ,-{Latex[width=4mm, length=6]}] (3pt,0pt) -- (80pt,-40pt) node[right]{Gestreutes Elektron} coordinate (d);
\node[starburst,inner color=orange,outer color=yellow,starburst points=11] at (0,0pt) {} coordinate (b);
\shade[ball color=green!70!white] (48pt, -23pt) circle (.2cm);
\draw [thick, ->] (5pt,0pt) -- (150pt,0pt) coordinate (a) node [right] {$x$};
\draw [black, thick ,-{Latex[length=4mm]},decorate, decoration={snake,amplitude=1mm,segment length=4mm,post length=4mm}, label=left:Incident photon] (5pt,0pt) -- (100pt,50pt) node[above]{Röntgenlicht mit Wellenlänge $\lambda'$} coordinate (c);
%\pic [black, draw, "$\theta$", angle eccentricity=0.8, angle radius=50pt] {angle = a--b--c};
% \pic [black, draw, "$\varphi$", angle eccentricity=0.8, angle radius=38pt] {angle = d--b--a};
\end{tikzpicture}
\caption{Schematische Darstellung des Compton-Effekts. Bildquelle: TikZ-Code angepasst von \cite{stackCompton}
\label{fig:cpt:compton}}
\end{figure}

\subsubsection{Herleitung der Compton-Formel mit dem Teilchenmodell}
\label{sec:cpt:herleitung}
\begin{minipage}{.7\linewidth}
Wir nehmen an, dass das Licht aus Teilchen, den \emph{Photonen}, besteht und dass eines davon mit dem Teilchen \emph{Elektron} einen elastischen Stoss vollführt, siehe Abbildung \ref{fig:cpt:compton}.
Unter dieser Annahme wollen wir zeigen, dass Gleichung \eqref{equ:cpt:ComptonFormel} gilt.

Wenn wir einen elastischen Stoss \glqq durchrechnen\grqq, dann greifen wir auf die zwei fundamentalen Erhaltungssätze der Physik zurück:
\begin{enumerate}[I)]\setlength\itemsep{0pt}
 \item Impulserhaltung
 \item Energieerhaltung
\end{enumerate}
Dazu führen wir die Bezeichungen an Rand ein und beginnen mit der {\bfseries Impulserhaltung}:
\end{minipage}
\hfill
\begin{minipage}{.19\linewidth}
\fbox{\parbox{\linewidth}{
    {\small
    {\bfseries Photon:}\\
    $p_i\colon$ initialer Impuls\\
    $E_i\colon$ initiale Energie\\
    $p_f\colon$ finaler Impuls\\
    $E_f\colon$ finaler Impuls\\~\\
    {\bfseries Elektron:}\\
    $p\colon$ Impuls\\
    $K\colon$ Gesamtenergie
    }}
}
\end{minipage}
\begin{align*}
 \text{Impulserhaltung in $x$-Richtung: }\qquad p_i &= p_f \cos\theta + p\cos\varphi\\
 \text{Impulserhaltung in $y$-Richtung: }\qquad 0 &= p_f \sin\theta - p\sin\varphi
\end{align*}
Umgestellt nach dem Elektronenanteil ergeben diese Gleichungen:
\begin{align}
 \text{Impulserhaltung in $x$-Richtung: }&&\qquad p_i - p_f \cos\theta &= p\cos\varphi\label{equ:cpt:ImpulsX}\\
 \text{Impulserhaltung in $y$-Richtung: }&&\qquad p_f \sin\theta &= p\sin\varphi\label{equ:cpt:ImpulsY}
\end{align}
Wir wollen nun beide Gleichungen zusammen verwenden, um den Winkel $\varphi$ zu eliminieren.
Dazu quadrieren wir die Gleichungen \eqref{equ:cpt:ImpulsX} und \eqref{equ:cpt:ImpulsY} erst und addieren sie dann.
So erhalten wir:
\begin{align}
 \left( p_i - p_f \cos\theta\right)^2 + \left( p_f \sin\theta \right)^2 = \left( p\cos\varphi \right)^2 + \left( p\sin\varphi \right)^2
 \label{equ:cpt:SquaredAdded}
\end{align}
Zuerst bearbeiten wir die rechte Seite von Gleichung \eqref{equ:cpt:SquaredAdded}, wobei wir im letzten Schritt den trigonometrischen Pythagoras anwenden:
\begin{align}
 \left( p\cos\varphi \right)^2 + \left( p\sin\varphi \right)^2 \quad=~ p^2 \cos^2\varphi + p^2 \sin^2\varphi
 \quad=~ p^2 \underbrace{\left( \cos^2\varphi + \sin^2\varphi \right)}_{=1} \quad=~ p^2\, .
 \label{equ:cpt:RHS}
\end{align}
Dann bearbeiten wir noch die linke Seite von Gleichung \eqref{equ:cpt:SquaredAdded}, indem wir mithilfe der binomischen Formel ausmultiplizieren:
\begin{align*}
 \left( p_i - p_f \cos\theta\right)^2 + \left( p_f \sin\theta \right)^2
 \quad=~ \underbrace{p_i^2 - 2p_i p_f \cos\theta + p_f^2 \cos^2\theta}_{\text{binomische Formel}} + p_f^2 \sin^2 \theta
\end{align*}
Die letzten beiden Summanden fassen wir noch mit dem trigonometrischen Pythagoras zusammen und erhalten
\begin{align}
 p_i^2 - 2p_i p_f \cos\theta + \underbrace{p_f^2 \cos^2\theta + p_f^2 \sin^2 \theta}_{=p_f^2 \left(\cos^2\theta + \sin^2 \theta\right) ~=~ p_f^2}
 \quad= p_i^2 - 2p_i p_f \cos\theta + p_f^2
 \label{equ:cpt:LHS}
\end{align}
Nun setzten wir die Gleichungen \eqref{equ:cpt:RHS} und \eqref{equ:cpt:LHS} wieder in \eqref{equ:cpt:SquaredAdded} ein und erhalten aus der Impulserhalten folgende Beziehung:
\begin{align}
 \fbox{\parbox{4cm}{$p_i^2 - 2p_i p_f \cos\theta + p_f^2 = p^2$}}
 \label{equ:cpt:Impulserhaltung}
\end{align}
D.h. wir haben nun schon einmal eine Formel, die Anfangs- und Endimpuls des Photons in Beziehung setzt.
Das ist gut.
Allerdings gibt es in der obigen Formel immer noch eine Abhängigkeit vom Elektronenimpuls $p$, die wir eliminieren müssen.

Zu diesem Zweck verwenden wir noch die {\bfseries Energieerhaltung}.\\
Da das gestreute Elektron eine sehr hohe Energie haben wird, müssen wir die Rechnung relativistisch durchführen und vor dem Stoss die Masseenergie des Elektrons berücksichtigen.
Deshalb erhalten wir folgende Energiebilanzierung für den Compton-Effekt:
\begin{align*}
  \underbrace{E_i + m_{0\el} c^2}_{\text{Energie vor dem Stoss}} = \underbrace{E_f + K}_{\text{Energie nach dem Stoss}} \qquad \Rightarrow\qquad E_f - E_f + m_{0\el} c^2 = K
\end{align*}
Da das Photon keine Ruhemasse besitzt, können wir für die Energien $E_i$ und $E_f$ direkt die kinetische Energie einsetzen:
\begin{align*}
  E_i = p_i c \qquad\text{und}\qquad E_f = p_f c
\end{align*}
Dies setzen wir in die Energiebilanzierung des Compton-Effekts ein:
\begin{align}
  c\left(  p_i - p_f \right) + m_{0\el} c^2 = K
  \label{equ:cpt:EnergieErhaltung}
\end{align}

Nun müssen wir noch die Gesamtenergie $K$ des Elektrons in Abhängigkeit seines Impulses $p$ einsetzen.
Durch die relativistische Massenzunahme des Elektrons gilt allgemein für die Gesamtenergie $K$ des Elektrons (Herleitung siehe Anhang \ref{sec:cpt:relativity} auf Seite \pageref{sec:cpt:relativity}):
\begin{align}
  K^2 = c^2 p^2 + \left( m_{0\el} c^2\right)^2\, .
  \label{equ:cpt:EnergieImpuls}
\end{align}

Wir wollen die allgemein gültige Gleichung \eqref{equ:cpt:EnergieImpuls} in die Energiebilanzierung \eqref{equ:cpt:EnergieErhaltung} des Compton-Effekts einsetzen.
Dafür müssen wir Gl. \eqref{equ:cpt:EnergieErhaltung} quadrieren:
\begin{align*}
 \underbrace{\left[ c\left(  p_i - p_f \right) + m_{0\el} c^2 \right]^2}_{\text{Gl. \ref{equ:cpt:EnergieErhaltung}}} \quad=\quad K^2 \quad=\quad \underbrace{ c^2 p^2 + \left( m_{0\el} c^2\right)^2}_{\text{Gl. \ref{equ:cpt:EnergieImpuls}}}
\end{align*}
Wir tun jetzt zwei Dinge gleichzeitig:
Wir verwenden auf der linken Seite die binomische Formel um das Quadrat in eine Summe zu verwandeln und setzen gleichzeitig auf der rechten Seite für $p^2$ die Formel von Gleichung \eqref{equ:cpt:Impulserhaltung} ein.
\begin{align*}
 c^2\left(  p_i - p_f \right)^2 + 2c^3\left(  p_i - p_f \right)m_{0\el} + m^2_{0\el} c^4 \quad&=\quad c^2 \left[ p_i^2 - 2p_i p_f \cos\theta + p_f^2 \right] + \left( m_{0\el} c^2\right)^2 \qquad &&\biggl|\, -\left( m_{0\el} c^2\right)^2\\~\\
 c^2\left(  p_i - p_f \right)^2 + 2c^3\left(  p_i - p_f \right)m_{0\el} \quad&=\quad c^2 \left[ p_i^2 - 2p_i p_f \cos\theta + p_f^2 \right] \qquad &&\biggl|\, :c^2\\~\\
 \left(  p_i - p_f \right)^2 + 2c\left(  p_i - p_f \right)m_{0\el} \quad&=\quad p_i^2 - 2p_i p_f \cos\theta + p_f^2\\~\\
\intertext{Jetzt wenden wir links die binomische Formel an.}
 p_i^2 - 2p_i p_f + p_f^2 + 2c\left(  p_i - p_f \right)m_{0\el} \quad&=\quad p_i^2 - 2p_i p_f \cos\theta + p_f^2 \qquad &&\biggl|\, -p_i^2 -p_f^2\\~\\
 - 2p_i p_f + 2c\left(  p_i - p_f \right)m_{0\el} \quad&=\quad - 2p_i p_f \cos\theta \qquad &&\biggl|\, :\left( 2p_i p_f \right)\\~\\
 -1 + m_{0\el}c\cdot \left( \frac{1}{p_f} - \frac{1}{p_i} \right)\quad&=\quad -\cos\theta&&\biggl|\, +1\\~\\
 m_{0\el}c\cdot \left( \frac{1}{p_f} - \frac{1}{p_i} \right)\quad&=\quad 1-\cos\theta&&\biggl|\, :(m_{0\el}c)\\~\\
 \frac{1}{p_f} - \frac{1}{p_i}\quad&=\quad \frac{1}{m_{0\el}c} \left( 1-\cos\theta \right)&&\biggl|\, \cdot\, h\\~\\
 \frac{h}{p_f} - \frac{h}{p_i}\quad&=\quad \frac{h}{m_{0\el}c} \left( 1-\cos\theta \right)
\end{align*}
Die rechte Seite der Gleichung sieht schon so aus wie die in Gleichung \eqref{equ:cpt:ComptonFormel}.
Im letzten Schritt müssen wir nur noch $\lambda = h/p$ einsetzen und erhalten:
\begin{align*}
 \lambda' - \lambda \quad&=\quad \frac{h}{m_{0\el}c} \left( 1-\cos\theta \right)
\end{align*}
Dies ist Gleichung \eqref{equ:cpt:ComptonFormel}.


\section{Photonen im Raum}


\newpage

\sloppy
\printbibliography[notkeyword=figure, notkeyword=misc]

\printbibliography[title={Quellenverzeichnis der Abbildungen}, keyword=figure, heading=secbib]
\printbibliography[title={Verzeichnis weiterer Quellen}, keyword=misc, heading=secbib]


\appendix

\clearpage
\pagenumbering{arabic}% resets `page` counter to 1
\renewcommand*{\thepage}{A\arabic{page}}

\thispagestyle{empty}
\etocdepthtag.toc{mtappendix}
\thispagestyle{empty}
\etocsettagdepth{mtchapter}{none}
\etocsettagdepth{mtappendix}{subsection}
\thispagestyle{empty}
\tableofcontents
\thispagestyle{empty}

\chapter{Einführung in die Quantenphysik}
\section{Quantelung des Lichts}
% \subsection{Hallwachs-Versuch}
% \subsection{Bestimmung von $h$ mit der Gegenfeldmethode}
% \subsection{Umkehrung des Fotoeffekts mit LEDs}
% \subsection{Kurzwellige Grenze von Röntgenlicht}
\subsection{Compton-Effekt}
\subsubsection{Zur Herleitung der Compton-Formel}
\paragraph{Exkurs Energie, Impuls und relativistische Massenzunahme.}
\label{sec:cpt:relativity}
Im Unterricht konntet ihr euch an folgende Formel zur relativistischen Massezunahme erinnern:
\begin{align}
  m = \frac{m_0}{\sqrt{1 - \frac{v^2}{c^2}}}
  \label{equ:cpt:Massenzunahme}
\end{align}
Wir wollen dies benutzen um die Gleichung
\begin{align}
  E = mc^2
  \label{equ:cpt:Emc2}
\end{align}
in Gleichung \eqref{equ:cpt:EnergieImpuls} auf Seite \pageref{equ:cpt:EnergieImpuls} zu überführen.
Zuerst aber quadrieren wir Gleichung \eqref{equ:cpt:Emc2}, spalten eine 1 ab und addieren eine gewinnbringende Null:
\begin{align*}
  E^2 \quad=~ m^2 c^4 \quad=~ m^2 c^4 \cdot 1 \quad=~ m^2 c^4 \cdot \left( \frac{v^2}{c^2} + 1 - \frac{v^2}{c^2} \right)
\end{align*}
Jetzt multiplizieren wir einmal aus:
\begin{align*}
 E^2 \quad=~ \frac{m^2 c^4 v^2}{c^2} + m^2 c^4 \cdot \left( 1 - \frac{v^2}{c^2} \right)
 \quad=~ m^2 c^2 v^2 + m^2 c^4 \cdot \left( 1 - \frac{v^2}{c^2} \right)\, .
\end{align*}
Hier erkennen wir schon den ersten Term von Gleichung \eqref{equ:cpt:EnergieImpuls}, denn $p = mv$ und deshalb: $m^2 c^2 v^2 = p^2 c^2$.
Also sehen wir schon hier:
\begin{align*}
 E^2 = p^2 c^2 + m^2 c^4 \cdot \left( 1 - \frac{v^2}{c^2} \right)\, .
\end{align*}
Im zweiten Summanden ersetzen wir nun $m$ durch die relativistische Formel in Gleichung \eqref{equ:cpt:Massenzunahme}:
\begin{align*}
 E^2 = p^2 c^2 + \frac{m_{0}^2 c^4}{1 - \frac{v^2}{c^2}} \cdot \left( 1 - \frac{v^2}{c^2} \right)\, .
\end{align*}
Kürzen ergibt:
\begin{align}
 E^2 \quad=\quad p^2 c^2 + m_{0}^2 c^4 \quad=\quad p^2 c^2 + \left( m_{0} c^2 \right)^2 \, .
\end{align}
Ersetzen wir noch $E$ durch $K$ und $m_0$ durch $m_{0\el}$, so erhalten wir Gleichung \eqref{equ:cpt:EnergieImpuls}.

\end{document}
